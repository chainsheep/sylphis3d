\documentclass[12pt, titlepage]{amsart}
\usepackage{graphicx}

\begin{document}
\title{Sylphis Shader Manual}
\author{Harry Kalogirou \\
email: harkal@gmx.net}
\maketitle
\tableofcontents
\pagebreak

%\begin{abstract}
%In Sylphis shading of polygons is described by small scripts called
%shaders. This document describes the syntax of the shader language
%along with the features it presents.
%\end{abstract}

\section{Introduction}
Sylphis has a very sophisticated rendering system that has many features
and all these features can be parameterized in many different ways. In
order to let the designers utilize all this power the Sylphis Shader
Language is used. In Sylphis you do not apply simple images on polygons
as textures. Shader scripts are assigned to every polygon to describe
the exact way to shade it. With this scripts you can apply textures,
bump and normal maps, specular maps, along with scaling, rotating,
blending, alpha masking them.

\subsection{Surface light response}
In Sylphis the way a surface responds to light is calculated for every pixel
that is drawn on the screen. That is to make perfect looking lighting of big
polygons. Having that as a fact it is obvious that it is proper to have surface
properties not assigned per polygon but per textel, something like what the texture map
does but for more properties than color. So in Sylphis we can have more than one
texture map per material each describing a different property.

\subsubsection{The color map}
This map stores he material color then it is ambient lit. This should not
have any shadows drawn on it. Just the color of the surface. Some soft shadows
might do some good but certainly not hard shadows.

\begin{figure}
\begin{center}
\includegraphics[scale=4]{color.eps} \\
\end{center}
\caption{A simple color map}
\label{figure:colormap}
\end{figure}

\subsubsection{The bump map}
This gray scale map stores he height of the material with black being low and white being high.
Using bump maps will make the surface look bumpy. This will be done by the engine by creating
bright and darker lit areas on the surface depending on the bump map and how the simulated
bumpy surface faces the light.

\subsubsection{The normal map}
The normal map has the same effect with the bump map, but it is created only by
certain programs. Every pixel of a normal map is actually the normal vector of the
surface at that place. This means that RGB is mapped to a (x,y,z) normal that is
used as the surface normal used in the lighting equations of the engine.
Most of the time normal maps create better results than bump maps since the
normal maps are more "relaxed" on the orientation of the normals.

Actually Sylphis internally always uses normal maps and when a bump map is supplied
it is converted to a normal map.

\begin{figure}
\begin{center}
\includegraphics[scale=1]{norm.eps} \\
\end{center}
\caption{A simple normal map}
\label{figure:normalmap}
\end{figure}

\subsubsection{The gloss map}
This map controls the color the surface reflects for when specular lit. This effectively
controls the shininess of the surface at every pixel. Wherever the gloss map has dark
colors the surface will look mat. Wherever the gloss map has bright colors the surface
will look shiny.

\begin{figure}
\begin{center}
\includegraphics[scale=1]{gloss.eps} \\
\end{center}
\caption{A simple gloss map}
\label{figure:glossmap}
\end{figure}

\subsection{Rendering samples}
\begin{figure}
\begin{center}
\includegraphics[scale=.45]{nobump_gimp.eps} \\
\end{center}
\caption{Simple texture applied on wall}
\label{figure:nobump}
\end{figure}

\begin{figure}
\begin{center}
\includegraphics[scale=.45]{withbump.eps} \\
\end{center}
\caption{Adding a normal map}
\label{figure:addnormalmap}
\end{figure}

\begin{figure}
\begin{center}
\includegraphics[scale=.45]{withgloss.eps} \\
\end{center}
\caption{Adding a gloss map}
\label{figure:addglossmap}
\end{figure}

\section{Shader syntax}
The Sylphis shaders have simple syntax. The main body of the shader
consists of a series of surface properties and rendering pass instructions
that are grouped within curly brackets \{\}.

section{The simplest script}
The simplest shading case in Sylphis is just applying a single texture
on a polygon. The Shader script to accomplish that is :

\begin{verbatim}
simple_shader {
    {
        map funny_face.jpg
    }
}
\end{verbatim}

The above shader has no surface parameters. It consists of only one pass
that has assigned the funny\_face.jpg texture. As we can see the shader
describes the rendering process as a series of rendering passes.

\section{Waveform functions}
\label{waveforms}
Some of the shader functions make use of \textit{Waveform functions} to modulate
certain values over time.

Syntax :
\begin{verbatim}
wave <waveform> <base> <scale> <phase> <freq>
\end{verbatim}

The \textit{waveform} can be one of :
\begin{itemize}
\item \textbf{sin} : Is a simply sine wave. Ranging from -1 to 1.
\item \textbf{triangle} : A wave with sharp ascent and then a sharp decay. Ranging for 0 to 1.
\item \textbf{square} : Simple switches between  -1 and 1.
\item \textbf{sawtooth} : A wave with a sharp ascent from 0 to 1 and then an instant decay to 0.
\item \textbf{inversesawtooth} : A wave with an instant ascent to 1 and then a sharp decay from 1 to 0.
\end{itemize}

The \textit{base} is the constant that is added to the wave form. For example the sin waveform
ranges from -1 to 1 with a \textit{base} of 0, but will range from 0 to 2, with a \textit{base}
of 1.

The \textit{scale} is the constant that will scale the wave form. For example the sin waveform
with a \textit{scale} value of 2 will range from -2 to 2. So \textit{base} and \textit{scale}
are applied on the original waveform function $g(t)$ like this :

\begin{center}
$f(t)=g(t)*scale + base$
\end{center}

The \textit{freq} if the waveform frequency in Hz and the \textit{phase} is the phase of the waveform.
This should be in the range 0 to 1. Other values will be normalized in that range.


\section{Rendering pass instructions}
Inside the block of a rendering pass is described the way the specific pass
will be rendered by the engine. The keywords for this are:

\begin{itemize}
\item map
\item animmap
\item normalmap
\item heightmap
\item glossmap
\item glossiness
\item rgbgen
\item blendfunc
\item alphafunc
\item tcmod
\item tcgen
\end{itemize}

\subsection{map}
Syntax :
\begin{verbatim}
map <image name>
\end{verbatim}

With this keyword we assign the base \textit{color map} of the surface.

\subsection{animmap}
Syntax :
\begin{verbatim}
animmap <fps> <image name 1> ... <image name n>
\end{verbatim}

This keyword works like the \textit{map} keyword but more than one
image can be assigned. This images are used in sequence and are
sequenced at \textit{fps} frames per second.

\subsection{normalmap}
Syntax :
\begin{verbatim}
normalmap <image_name>
\end{verbatim}

With this keyword the \textit{normal map} of the surface is assigned.
The image file specified as \textit{image\_name} is an image with normals
coded in as colors in it.

\subsection{heightmap}
Syntax :
\begin{verbatim}
heightmap <image_name> [hardness]
\end{verbatim}

With this keyword we specify the \textit{height map} to be used to
generate the \textit{normal map} for the surface. Optionally we can
use the \textit{hardness} value to control the bumpness amount. The
default value is 1.0.

\subsection{glossmap}
Syntax :
\begin{verbatim}
glossmap <image_name>
\end{verbatim}

With this keyword we specify the color the surface emits at specular
highlights. This can be a color image.

\subsection{glossness}
Syntax :
\begin{verbatim}
glossness <amount>
\end{verbatim}

This keyword is used when the surface is constantly glossy over the
surface. The \textit{amount} is a value from 0.0 to 1.0, with 0.0
meaning totally mat surface and 1.0 a very glossy surface.

\subsection{rgbgen}
Syntax :
\begin{verbatim}
rgbgen <type> [additional parameters]
\end{verbatim}

This instruction controls the way the coloring of the surface is done.
The \textit{type} can be \textit{color} or a waveform function (Section \ref{waveforms}).

\subsubsection{color}
Syntax :
\begin{verbatim}
color <r> <g> <b>
\end{verbatim}

Sets the color of the surface to \textit{(r, g, b)}. The color components
range from 0.0 to 1.0

\subsection{blendfunc}
Syntax :
\begin{verbatim}
blendfunc <type>
\end{verbatim}

This instruction affects the way the current pass is blended with the
previous pass and effectively with the framebuffer. The \textit{type}
can be :

\begin{itemize}
\item \textit{blend}
\item \textit{filter}
\item \textit{add}
\end{itemize}

\subsection{alphafunc}
Syntax :
\begin{verbatim}
alphafunc <type>
\end{verbatim}

This instruction sets the alpha test for the surface. The type can be
\textit{gt0} or \textit{gt128}.

\subsection{tcmod}
Syntax :
\begin{verbatim}
tcmod <type> [additional parameters]
\end{verbatim}

This instruction controls the way the original texturing coordinates
of the surface are modified. The \textit{type} can be
\textit{scale, rotate, scroll, stretch, transform} and \textit{turb}.

\subsubsection{scale}
Syntax :
\begin{verbatim}
scale <s> <t>
\end{verbatim}

Scales the texturing coordinates by \textit{s} and \textit{t}

\subsubsection{rotate}
Syntax :
\begin{verbatim}
rotate <r_speed>
\end{verbatim}

Rotates the texture image on the surface with \textit{r\_speed} speed.

\subsubsection{scroll}
Syntax :
\begin{verbatim}
scroll <s_speed> <t_speed>
\end{verbatim}

scrolls the texture image on the surface with \textit{s\_speed} and
\textit{t\_speed} speed.

\subsubsection{stretch}
Syntax :
\begin{verbatim}
stretch <waveform function>
\end{verbatim}

Stretches the texture based on a periodical waveform function (Section \ref{waveforms}).

\subsubsection{transform}
Syntax :
\begin{verbatim}
transform <m00> <m01> <m10> <m11> <t0> <t1>
\end{verbatim}

Transforms texture coordinates like this :

\begin{center}
$s' = s * m00 + t * m10 + t0$

$t' = t * m01 + s * m11 + t1$
\end{center}

\subsection{tcgen}
Syntax :
\begin{verbatim}
tcgen <type>
\end{verbatim}

This instruction makes the engine generate new texturing coordinates for
the surface. The \textit{type} can be \textit{environment}.

The \textit{environment} texturing coordinate generation causes the
texture to look like a reflection. Also known as environment mapping.

\section{Surface properties}
The surface properties keywords are
\begin{itemize}
\item surfaceparm
\item nomipmaps
\item deformvertices
\item sky
\item qer\_editorimage
\end{itemize}

\subsection{surfaceparm}
With this surface parameter we can change certain surface properties.
The available keywords are \textit{trans} and \textit{noshadows}.
The keyword \textit{trans} instructs that the surface is transparent. This
means that it is possible to look through a surface with that shader
assigned. The keyword \textit{noshadows} on the other hand instructs
the engine not to generate shadows from surfaces with that shader. This
means that light will not be blocked by these surfaces.

Below is an example shader below causes surfaces not to cast shadows.

\begin{verbatim}
magic_wall {
    surfaceparm noshadows
    {
        map bricks.jpg
    }
}
\end{verbatim}

\subsection{nomipmaps}
This keyword instructs the engine not generate mipmaps
\footnote{Mipmaps are progressive maps that are generated from the given
texture map by halfing the size of it every time. For example from an
128x128 texture the engine generates mipmaps with sizes 64x64, 32x32,
16x16, 8x8, 4x4, 2x2, 1x1. Smaller mipmaps are used at greater distances
to avoid aliasing}
for the textures used in this shader. This useful for cases were the
same texture must be used at all distances from the camera.

\subsection{deformvertices}
The \textit{deformvertices} keyword affects the actual position and shape of the surface.
The vertices of the surface are modified with this shader keyword. Multiple
deform types can be used on a single shader.

Syntax :
\begin{verbatim}
deformvertices <type> [additional parameters]
\end{verbatim}

The \textit{type} keyword can be one of these :

\begin{itemize}
\item wave
\item normal
\item move
\end{itemize}

\subsubsection{wave}
Syntax :
\begin{verbatim}
wave <spread> <waveform function>
\end{verbatim}

This will give a rippling effect on the surface. Vertices will move along their normal's
direction using the periodic waveform function with different phases. The \textit{spread}
keyword suggests the "spread" of the wave.

\subsubsection{normal}
Syntax :
\begin{verbatim}
normal <spread> <waveform function>
\end{verbatim}

This vertex deform keyword will work like the \textit{wave} but will affect the normals and
not the vertices. This will affect the way the surface will be lit and reflect light.

\subsubsection{}
Syntax :
\begin{verbatim}
move <x> <y> <z> <waveform function>
\end{verbatim}

This vertex deform keyword will make the surface attached to appear to move altogether.
The \textit{x}, \textit{y} and \textit{z} are the max distance that the surface will
move on relatively to the "rest" position of the surface. The waveform function modulates
the \textit{x}, \textit{y} and \textit{z} so the surface will move in a periodical fashion.

Note that the actual surface doesn't move, it just looks like it moves.

\subsection{sky}
The keyword \textit{sky} makes this shader a sky-shader. This is a
special kind of header and is explained in detail in an other part
of this document.

\subsection{qer\_editorimage}
This keyword is not intended for the engine itself but for the editing tool.
The syntax is :

\begin{verbatim}
qer_editorimage <texture name>
\end{verbatim}

The \textit{texture name} is the texture to be used in the editor.

\end{document}
